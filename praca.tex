\documentclass[a4paper,12pt]{article}
\usepackage{indentfirst}
\usepackage[utf8]{inputenc}
\usepackage{polski}
\usepackage[margin=3.0cm]{geometry}
\linespread{1.3}            % interlinia 1.5
\frenchspacing              % brak większego odstępu po kropce
\sloppy                     % lepsze łamanie linii
\hyphenation{}
\author{Rafał Piotrowski}
\title{Interaktywność jako podstawowa cecha net art}
\begin{document}

\maketitle

%\tableofcontents

%\section{Wstęp}
%Wstęp.

\section{Co to jest net art}
%Historia nazwy net.art i przekształcenie jej do nazwy net art.
%Inne konkurencyjne nazwy: sztuka internetu, sztuka sieci.
%Szczegółowe rozróżnienia, które rodzą kolejne pojęcia takie jak:
%www art, browser art, ASCII art, itp. Wskazanie na problem
%kształtującego się nazewnictwa związany, moim zdaniem, z krótkim
%okresem występowania zjawiska jakim jest sztuka internetu.
%Przedstawienie kilku definicji net artu i moich opinii na ich temat.
Historia nazwy net art rozpoczęła się w roku 1995, kiedy to Vuk Cosic
otrzymał anonimową wiadomość e-mail, której jedynym czytelnym fragmentem
były słowa ,,Net.Art''. Później okazało się, że e-mail był zaszyfrowany,
a po odszyfrowaniu, fragment ,,Net.Art'' należał do dwóch różnych zdań
oddzielonych kropką. Nie ma niestety żadnego dowodu na potwierdzenie
tej opowieści, ponieważ adresat twierdzi, że jedyna kopia tego listu
została utracona w wyniku awarii dysku twardego. Bez względu na to
czy historia ta jest prawdziwa czy nie, nazwa zyskała powszechne uznanie.
Nazwa ta bardzo dobrze nadaje się na określenie sztuki związanej
nieodłącznie z komputerami i Internetem, ponieważ kojarzy się
z nazwą domenową lub z nazwą pliku.

W kolejnych latach pojawiły się nowe określenia i rozróżnienia.
Oprócz nazwy net.art zaczęła funkcjonować wersja bez kropki - net art
oraz inne takie jak: internet art (sztuka internetu), www art, web art,
net-specific art, a także określenia rozdzielające dzieła net-art
na różne kategorie, na przykład: browser art, ASCII art, e-mail art, itp.

W tej pracy zamierzam używać zamiennie nazw sztuka internetu i net art
na określenie tego co zawierają wszystkie wyżej wymienione terminy
i określenia. Zanim zajmę się omówieniem jakichkolwiek rzeczy związanych
z zagadnieniem sztuki internetu, chciałbym przedstawić definicje
tego terminu i odnieść się do nich.
%definicja ewy
%definicja ryszarda
%definicja angielskiej wikipedii

Krótki okres istnienia sztuki internetu powoduje moim zdaniem problemy
z terminologią. Brak nam jeszcze odpowiedniego dystansu do tego zjawiska
i czasu na wykształcenie się aparatu pojęciowego, który w dużej mierze
musi zostać wymyślony z powodu nowości technologii i sposobu oddziaływania
dzieł net art na odbiorcę.

\section{Przykłady dzieł net art}
Kilka przykładów dzieł net art, zarówno wczesnych jak i późniejszych.
Chcę tutaj przybliżyć czytelnikowi temat zagadnienia net art poprzez
konkretne przykłady.

\section{Wpływy i historia}
Omówienie wpływów wcześniejszych ruchów na powstanie net art.
Wskazać chcę przede wszystkim na:
\begin{itemize}
\item mail art (sztuka poczty (nieelektronicznej))
\item konceptualizm
\item sztuka kinetyczna
\item performance
\item sztuka wideo
\item wielka awangarda
\end{itemize}
Dla każdego ,,kierunku'' chcę wskazać co wiąże go z net art,
dlaczego jest uważany za ,,prekursora'' net art.

\section{Cechy net art}
Zebranie cech dzieł net art wymienionych
w poprzednim rozdziale oraz dodanie (jeśli się uda) kolejnych.
Podsumowanie, wskazanie na najważniejsze.

\section{Interaktywność}
Szersze omówienie pojęcia interaktywności, różne możliwe
definicje, podziały i rozróżnienia.

\section{Czy net art jest interaktywny}
Konfrontacja net art i różnych definicji interaktywności.
Próba odpowiedzi napytanie czy każde dzieło net art jest
interaktywne.

%\section{Zakończenie}
%Zakończenie.

\begin{thebibliography}{6}
\bibitem{EWA}
  Ewa Wójtowicz:
  \textit{Net art},
  Wydawnictwo Rabid, Kraków 2008
\bibitem{SZT-I}
  Ryszard W. Kluszczyński:
  \textit{Sztuka interaktywna},
  Wydawnictwa Akademickie i Profesjonalne,
  Warszawa 2010
\bibitem{E-I-AE}
  Maria Gołaszewska:
  \textit{Estetyka i antyestetyka},
  Wiedza Powszechna, Warszawa 1984
\bibitem{CYBERKULTURA}
  Ryszard W. Kluszczyński:
  \textit{Społeczeństwo informacyjne. Cyberkultura. Sztuka multimediów},
  Wydawnictwa Rabid, Kraków 2002
\bibitem{ESTETYKA-WIRTUALNOSCI}
  praca wspólna pod red. Michała Ostrowickiego:
  \textit{Estetyka Wirtualności},
  Towarzystwo Autorów i Wydawców Prac Naukowych UNIVERSITAS,
  Kraków 2005
\bibitem{MCLUHAN}
  Marshall McLuhan:
  \textit{Wybór tekstów},
  red. Eric McLuhan, Frank Zingrone,
  przeł. Ewa Różalska, Jacek M. Stokłosa
  Zysk i S-ka Wydawnictwo, Poznań 2001

%tu nie ma praktycznie nic na temat net art
%\bibitem{OBRAZY-NA-WOLNOSCI}
%  Ryszard W. Kluszczyński:
%  \textit{Obrazy na wolności. Studia z zakresu sztuk medialnych w Polsce.},
%  Instytut Kultury, Warszawa 1998

\end{thebibliography}

\end{document}

% vi:set encoding=utf8:
