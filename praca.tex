\documentclass[a4paper,12pt]{article}
\usepackage{indentfirst}
\usepackage[utf8]{inputenc}
\usepackage{polski}
\usepackage[margin=3.0cm]{geometry}
%\linespread{1.3}            % interlinia 1.5
\frenchspacing              % brak większego odstępu po kropce
\sloppy                     % lepsze łamanie linii
\hyphenation{}
\author{Rafał Piotrowski}
\title{Net art a interaktywność}
\begin{document}

\maketitle

\tableofcontents

\section{Wstęp}
Wstęp.

\section{Co to jest net art}
Historia nazwy net.art, przekształcenie do nazwy net art,
inne konkurencyjne nazwy (sztuka internetu, sztuka sieci).
Szczególowe rozróznienia, które rodzą kolejne pojęcia,
takie jak www art, browser art, ascii art, itp.
Ogólny problem kształcującego
się nazewnictwa związany moim zdaniem z krótkim okresem występowania
zjawiska jakim jest sztuka internetu. Różne definicje (Ewy oraz Ryszarda)
oraz moje odniesienie się do tego.

\section{Przykłady dzieł net art}
Tak jak tytuł wskazuje.

\section{Wpływy i historia}
Omówienie wpływów wcześniejszych ruchów na powstanie net art.
Wzkazać chcę przede wszystkim na:
\begin{itemize}
\item mail art (sztuka poczty (nieelektronicznej))
\item konceptualizm
\item sztuka kinetyczna
\item performance
\item sztuka wideo
\item wielka awangarda
\end{itemize}
Dla każdego "kierunku" chcę wskazać co wiąże go z net art,
dlaczego jest uważany za "prekursora" net-art.

\section{Cechy net art}
Wymienienie cech dzieł net art, po części przytoczonych już
w poprzednim rozdziale. Zebranie cech dzieł net art wymienionych
w poprzednim rozdziale oraz dodanie (jeśli się uda) kolejnych.
Podsumowanie, wskazanie na najważniejsze. 

\section{Interaktywność}
Szersze omówienie pojęcia interaktywności, różne możliwe
definicje, podziały i rozróżnienia.

\section{Czy net art jest interaktywny}
Konfrontacja net art i różnych definicji interaktywności.
Próba odpowiedzi napytanie czy każde dzieło net art jest
interkatywne.

\section{Zakończenie}
Zakończenie.

\begin{thebibliography}{6}
\bibitem{EWA}
  Ewa Wójtowicz:
  \textit{Net art},
  Wydawnictwo Rabid, Kraków 2008
\bibitem{SZT-I}
  Ryszard W. Kluszczyński:
  \textit{Sztuka interaktywna},
  Wydawnictwa Akademickie i Profesjonalne,
  Warszawa 2010
\bibitem{E-I-AE}
  Maria Gołaszewska:
  \textit{Estetyka i antyestetyka},
  Wiedza Powszechna, Warszawa 1984
\bibitem{CYBERKULTURA}
  Ryszard W. Kluszczyński:
  \textit{Społeczeństwo informacyjne. Cyberkultura. Sztuka multimediów},
  Wydawnictwa Rabid, Kraków 2002
\bibitem{ESTETYKA-WIRTUALNOSCI}
  praca wspólna pod red. Michała Ostrowickiego:
  \textit{Estetyka Wirtualności},
  Towarzystwo Autorów i Wydawców Prac Naukowych UNIVERSITAS,
  Kraków 2005
\bibitem{MCLUHAN}
  Marshall McLuhan:
  \textit{Wybór tekstów},
  red. Eric McLuhan, Frank Zingrone,
  przeł. Ewa Różalska, Jacek M. Stokłosa
  Zysk i S-ka Wydawnictwo, Poznań 2001

%tu nie ma praktycznie nic na temat net art
%\bibitem{OBRAZY-NA-WOLNOSCI}
%  Ryszard W. Kluszczyński:
%  \textit{Obrazy na wolności. Studia z zakresu sztuk medialnych w Polsce.},
%  Instytut Kultury, Warszawa 1998

\end{thebibliography}

\end{document}

% vi:set encoding=utf8:
