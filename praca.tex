\documentclass[a4paper,12pt]{article}
\usepackage{indentfirst}
\usepackage[utf8]{inputenc}
\usepackage{polski}
\usepackage[margin=3.0cm]{geometry}
\linespread{1.3}            % interlinia 1.5
\frenchspacing              % brak większego odstępu po kropce
\sloppy                     % lepsze łamanie linii
\hyphenation{}
\author{Rafał Piotrowski}
\title{Interaktywność jako podstawowa cecha net art}
\begin{document}

\maketitle

%\tableofcontents

\section{Wstęp}
Praca ta ma celu potwierdzić tezę o interaktywności
jako postawowej cesze net art. W pierwszych rozdziałach omówione
zostają nazwy używane na określenie tego rodzaju sztuki,
oraz przykłady dzieł net art. Dalej omówiony jest kontekst historyczny,
wpływy wcześniejszych kierunków i realizacji artystycznych na powstanie
i cechy sztuki internetu, które dokładnie omówione są w osobym rozdziale.
Dwie ostatnie części pracy poświęcone są na analizę różnych rodzajów
interaktywności i potwierdzenie tezy zawartej w temacie pracy.
%miłego czytania życzę ;)

\section{Co to jest net art?}
%Historia nazwy net.art i przekształcenie jej do nazwy net art.
%Inne konkurencyjne nazwy: sztuka internetu, sztuka sieci.
%Szczegółowe rozróżnienia, które rodzą kolejne pojęcia takie jak:
%www art, browser art, ASCII art, itp. Wskazanie na problem
%kształtującego się nazewnictwa związany, moim zdaniem, z krótkim
%okresem występowania zjawiska jakim jest sztuka internetu.
%Przedstawienie kilku definicji net artu i moich opinii na ich temat.

%historia nazwy
Historia nazwy net art rozpoczęła się w roku 1995, kiedy to Vuk Cosic
otrzymał anonimową wiadomość e-mail, której jedynym czytelnym fragmentem
były słowa ,,Net.Art''. Później okazało się, że e-mail był zaszyfrowany,
a po odszyfrowaniu, fragment ,,Net.Art'' należał do dwóch różnych zdań
oddzielonych kropką. Nie ma niestety żadnego dowodu na potwierdzenie
tej opowieści, ponieważ adresat twierdzi, że jedyna kopia tego listu
została utracona w wyniku awarii dysku twardego. Bez względu na to
czy historia ta jest prawdziwa czy nie, nazwa zyskała powszechne uznanie.
Nazwa ta bardzo dobrze nadaje się na określenie sztuki związanej
nieodłącznie z komputerami i Internetem, ponieważ kojarzy się
z nazwą domenową lub z nazwą pliku.

%inne nazwy
W kolejnych latach pojawiły się nowe określenia i rozróżnienia.
Oprócz nazwy net.art zaczęła funkcjonować wersja bez kropki - net art
oraz inne takie jak: internet art (sztuka internetu), www art, web art,
net-specific art, a także określenia rozdzielające dzieła net-art
na różne kategorie, na przykład: browser art, ASCII art, e-mail art, itp.

%ja używam dwóch nazw
Krótki okres istnienia sztuki internetu powoduje moim zdaniem problemy
z terminologią. Brak nam jeszcze odpowiedniego dystansu do tego zjawiska
i czasu na wykształcenie się aparatu pojęciowego, który w dużej mierze
musi zostać wymyślony z powodu nowości technologii i sposobu oddziaływania
dzieł net art na odbiorcę. Powstało więc dużo określeń, z których żadne
na razie nie zwyciężyło w stu procentach.
W tej pracy zamierzam używać zamiennie nazw sztuka internetu i net art
na określenie tego co zawierają wszystkie wyżej wymienione terminy
i określenia. Zanim zajmę się omówieniem jakichkolwiek rzeczy związanych
z zagadnieniem sztuki internetu, chciałbym przedstawić definicje
tego terminu i odnieść się do nich.
%definicje
Jedną z pierwszym definicji
w polskiej literaturze fachowej przedstawił Ryszard Kluszczyński w roku 2002:
%Ryszard
\textit{Net art jest alternatywnym układem sztuki, który funkcjonuje
poza rynkiem określanym przez system galeryjny. Dąży do przełamania
izolacji jednostek, czyniąc z twórczej samoekspresji, kontaktu i globalnej,
wielokierunkowej komunikacji, podstawowe jakości sztuki. Otwartość,
powszechność, bezgraniczność, to cechy net art; akcentowanie wolności,
anarchiczność oraz zainteresowania społeczne i polityczne, to charakteystyczne
atrybuty postawy artystów realizujących swoje twórcze projekty
w cyberprzestrzeni.}
Zastrzegł, iż jest to definicja ogólna i wstępna, przewidując, że refleksja
nad sztuką Internetu będzie musiała zmieniać się wraz ze swoim przedmiotem.
Net art zmienia się szybko, równie szybko co technologia komputerowa,
nie mniej jednak ogólność definicji Ryszarda Kluszczyńskiego powoduje
jej aktualność. Według mnie, brak w niej podkreślenia związku net artu
z Internetem i technologią komputerową oraz niepotrzebny nacisk na
funkcjonowanie poza systemem galeryjnym, co przecież z biegiem czasu
może się zmienić.
Nieco późniejszą definicję sformułowała Ewa Wójtowicz. Określa ona net art
jako %Ewa
\textit{niematerialną, interaktywną sztukę, której podstawowym
medium i kontekstem jest Internet. Tematyka net artu najczęściej dotyczy
Internetu i komunikacji sieciowej oraz technologii. Projekt net art ma
hipertekstową, nielinearną budowę i zakłada aktywny odbiór w postaci
wielokierunkowej nawigacji. Oprócz wizualnej postaci dzieła, ważny jest
proces interaktywnej komunikacji. (...) Net artyści unikają definicji
i zacierają granice między sztuką a przestrzenią Sieci}.
Jest to definicja dużo bardziej konkretna. Ścisły związek net artu
z Internetem i technologią jest bezpośrednio wyrażony wraz z kilkoma
istotnym cechami sztuki Internetu. Definicja pochodzi z roku 2008
i na razie (do roku 2010) w pełni się, według mnie, broni.
Na koniec chciałbym przytoczyć jeszcze obecną definicję net artu znajdującą
się w wikipedii. Znajduje się ona pod hasłem ,,Sztuka internetu'' i ma
następującą postać: %Wikipedia
\textit{Sztuka, której głównym
medium, a najczęściej również tematem jest Internet. Istnieje tu podobieństwo
do sztuki wideo używającej przekazu wizualnego jako nośnika i jako tematu,
lecz z większym jeszcze naciskiem na temat niż na ów nośnik. Cytując definicję
stworzoną przez Steve'a Dietza, byłego kustosza działu nowych mediów
w Centrum Sztuki Walkera w Minneapolis: internetowe projekty artystyczne
to ,,projekty artystyczne, dla których sieć jest zarówno wystarczającym
jak i koniecznym warunkiem do zaistnienia.'' (...) Sztuka internetu może
przyjąć konkretną formę w postaci artystycznych stron www, sztuki w e-mail,
artystycznego oprogramowania internetowego, bazujących na internecie lub
połączonych w sieć instalacjach, online wideo, pracach radiowych lub audio,
sieciowych występach i instalacjach lub występach offline.}
Najbardziej odpowiadającą dla mnie jest definicja pani Ewy Wójtowicz,
która zawiera wszystkie najważniejsz cechy net art. Bardzo trafne jest
także określenie cytowane w wikipedii - net art to projekty, dla których
sieć jest zarówno wystarczającym jak i koniecznym warunkiem do zaistnienia.

\section{Przykłady dzieł net art}
%Kilka przykładów dzieł net art, zarówno wczesnych jak i późniejszych.
%Chcę tutaj przybliżyć czytelnikowi temat zagadnienia net art poprzez
%konkretne przykłady.

%wstęp
Przedstawienie kilku przykładów zrealizowanych dzieł net art z pewnością
przybliży zagadnienie w najlepszy sposób. w tym miejscu chciałbym
opisać kilka dzieł, zarówno z początkowej fazy zjawiska jak i z dojrzałego
etapu, który trwa do dziś.

%The World's First Collaborative Sentence
Prawdopodobnie pierwszym, a na pewno jednym z pierwszych przedsięwzięć,
które mogą zosać nazwane dziełem net art jest strona Davisa Douglasa
o nazwie \textit{The World's First Collaborative Sentence}. ble ble ble.

\section{Wpływy i historia}
Omówienie wpływów wcześniejszych ruchów na powstanie net art.
Wskazać chcę przede wszystkim na:
\begin{itemize}
\item mail art (sztuka poczty (nieelektronicznej))
\item konceptualizm
\item sztuka kinetyczna
\item performance
\item sztuka wideo
\item wielka awangarda
\end{itemize}
Dla każdego ,,kierunku'' chcę wskazać co wiąże go z net art,
dlaczego jest uważany za ,,prekursora'' net art.

\section{Cechy net art}
Zebranie cech dzieł net art wymienionych
w poprzednim rozdziale oraz dodanie (jeśli się uda) kolejnych.
Podsumowanie, wskazanie na najważniejsze.

\section{Interaktywność}
Szersze omówienie pojęcia interaktywności, różne możliwe
definicje, podziały i rozróżnienia.

\section{Czy net art jest interaktywny}
Konfrontacja net art i różnych definicji interaktywności.
Próba odpowiedzi napytanie czy każde dzieło net art jest
interaktywne.

%\section{Zakończenie}
%Zakończenie.

\begin{thebibliography}{6}
\bibitem{EWA}
  Ewa Wójtowicz:
  \textit{Net art},
  Wydawnictwo Rabid, Kraków 2008
\bibitem{SZT-I}
  Ryszard W. Kluszczyński:
  \textit{Sztuka interaktywna},
  Wydawnictwa Akademickie i Profesjonalne,
  Warszawa 2010
\bibitem{MARYLA}
  praca wspólna pod red. Maryli Hopfinger:
  \textit{Nowe media w komunikacji społecznej w XX wieku}
  Oficyna wydawnicza,
  Warszawa 2002
\bibitem{E-I-AE}
  Maria Gołaszewska:
  \textit{Estetyka i antyestetyka},
  Wiedza Powszechna, Warszawa 1984
\bibitem{CYBERKULTURA}
  Ryszard W. Kluszczyński:
  \textit{Społeczeństwo informacyjne. Cyberkultura. Sztuka multimediów},
  Wydawnictwa Rabid, Kraków 2002
\bibitem{ESTETYKA-WIRTUALNOSCI}
  praca wspólna pod red. Michała Ostrowickiego:
  \textit{Estetyka Wirtualności},
  Towarzystwo Autorów i Wydawców Prac Naukowych UNIVERSITAS,
  Kraków 2005
\bibitem{MCLUHAN}
  Marshall McLuhan:
  \textit{Wybór tekstów},
  red. Eric McLuhan, Frank Zingrone,
  przeł. Ewa Różalska, Jacek M. Stokłosa
  Zysk i S-ka Wydawnictwo, Poznań 2001

%tu nie ma praktycznie nic na temat net art
%\bibitem{OBRAZY-NA-WOLNOSCI}
%  Ryszard W. Kluszczyński:
%  \textit{Obrazy na wolności. Studia z zakresu sztuk medialnych w Polsce.},
%  Instytut Kultury, Warszawa 1998

\end{thebibliography}

\end{document}

% vi:set encoding=utf8:
