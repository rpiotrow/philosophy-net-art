%Jak zrobić praca.rtf:
%1) odkomentować linijkę z \usepackage[margin=3.cm]{geometry}
%2) make && latex2rtf praca && ooffice praca.rtf
%3) Select all && Format > Paragraph > Indents & Spacing > Line spacing na 1.5
%4) Prawy > Edit paragraph style > rozmiar czcionki na 12pt
%Uwaga: wszystkie \footnote{ zaczynać spacją} w ps nie ma różnicy, w rtf
%       wygląda dużo lepiej


%\documentclass[a4paper,12pt,twoside,openright]{report}
\documentclass[a4paper,12pt,twoside]{article}
\usepackage{indentfirst}    % nie działa w rtf
\usepackage[utf8]{inputenc}
\usepackage{polski}
%\usepackage[margin=3.0cm]{geometry}  % to być musi włączone dla rtf
\linespread{1.3}            % interlinia 1.5 % nie działa w rtf
\frenchspacing              % brak większego odstępu po kropce
\sloppy                     % lepsze łamanie linii
\setlength{\oddsidemargin}{0.6cm}  
\setlength{\evensidemargin}{-0.6cm}
\setlength{\marginparwidth}{0in}
\setlength{\marginparsep}{0pt}
\setlength{\voffset}{20pt}
\setlength{\hoffset}{19pt}
\setlength{\topmargin}{0in}
\setlength{\headheight}{0pt}
\setlength{\headsep}{0pt}
\setlength{\topskip}{0pt}
\setlength{\footskip}{50pt}
\setlength{\textwidth}{\paperwidth}
\addtolength{\textwidth}{-2.5in}
\setlength{\textheight}{\paperheight}
\addtolength{\textheight}{-3in}
\hyphenation{}
\author{Rafał Piotrowski}
\title{Interaktywność jako podstawowa cecha net art}
\begin{document}

\maketitle

%\tableofcontents

%\chapter{Wstęp}
\section{Wstęp}
Praca ta ma celu potwierdzić tezę o interaktywności
jako podstawowej cesze net art. W pierwszych rozdziałach omówione
zostają nazwy używane na określenie tego rodzaju sztuki
oraz przykłady dzieł net art. Dalej omówiony jest kontekst historyczny,
wpływy wcześniejszych kierunków i realizacji artystycznych na powstanie
i cechy sztuki Internetu. Cechy te, dokładnie omówione są w osobym rozdziale.
Dwie ostatnie części pracy poświęcone są analizie różnych rodzajów
interaktywności i potwierdzeniu tezy zawartej w temacie pracy.
%miłego czytania życzę ;)

%\chapter{Co to jest net art?}
\section{Co to jest net art?}
%Historia nazwy net.art i przekształcenie jej do nazwy net art.
%Inne konkurencyjne nazwy: sztuka internetu, sztuka sieci.
%Szczegółowe rozróżnienia, które rodzą kolejne pojęcia takie jak:
%www art, browser art, ASCII art, itp. Wskazanie na problem
%kształtującego się nazewnictwa związany, moim zdaniem, z krótkim
%okresem występowania zjawiska jakim jest sztuka internetu.
%Przedstawienie kilku definicji net artu i moich opinii na ich temat.

%historia nazwy
Historia nazwy net art rozpoczęła się w roku 1995, kiedy to Vuk Cosic
otrzymał anonimową wiadomość e-mail, której jedynym czytelnym fragmentem
były słowa ,,Net.Art''. Później okazało się, że e-mail był zaszyfrowany,
a po odszyfrowaniu, fragment ,,Net.Art'' należał do dwóch różnych zdań
oddzielonych kropką. Nie ma niestety żadnego dowodu na potwierdzenie
tej opowieści, ponieważ adresat twierdzi, że jedyna kopia tego listu
została utracona w wyniku awarii dysku twardego. Bez względu na to
czy historia ta jest prawdziwa czy nie, nazwa zyskała powszechne uznanie.
Bardzo dobrze nadaje się ona na określenie sztuki związanej
nieodłącznie z komputerami i Internetem, ponieważ kojarzy się
z nazwą domenową lub z nazwą pliku.

%inne nazwy
W kolejnych latach pojawiły się nowe określenia i rozróżnienia.
Oprócz nazwy net.art zaczęła funkcjonować wersja bez kropki -- net art --
oraz inne takie jak: Internet art (sztuka Internetu), www art, web art,
net-specific art, a także określenia rozdzielające dzieła net-art
na różne kategorie, na przykład: browser art, ASCII art, e-mail art, itp.

%ja używam dwóch nazw
Krótki okres istnienia sztuki Internetu powoduje moim zdaniem problemy
z terminologią. Brak nam jeszcze odpowiedniego dystansu do tego zjawiska
i czasu na wykształcenie się aparatu pojęciowego, który w dużej mierze
musi zostać wymyślony z powodu nowości technologii i sposobu oddziaływania
dzieł net art na odbiorcę. Powstało więc dużo określeń, z których żadne
na razie nie zwyciężyło w stu procentach.
W tej pracy zamierzam używać zamiennie nazw sztuka Internetu i net art
na określenie tego co zawierają wszystkie wyżej wymienione terminy
i określenia. Zanim zajmę się omówieniem jakichkolwiek rzeczy związanych
z zagadnieniem, chciałbym przedstawić definicje i odnieść się do nich.
%definicje
Jedną z pierwszym definicji net artu
w polskiej literaturze fachowej przedstawił Ryszard Kluszczyński w roku 2002:
%Ryszard
\textit{Net art jest alternatywnym układem sztuki, który funkcjonuje
poza rynkiem określanym przez system galeryjny. Dąży do przełamania
izolacji jednostek, czyniąc z twórczej samoekspresji, kontaktu i globalnej,
wielokierunkowej komunikacji, podstawowe jakości sztuki. Otwartość,
powszechność, bezgraniczność, to cechy net art; akcentowanie wolności,
anarchiczność oraz zainteresowania społeczne i polityczne, to charakteystyczne
atrybuty postawy artystów realizujących swoje twórcze projekty
w cyberprzestrzeni}\footnote{ Ryszard Kluszczyński,
\textit{Społeczeństwo informacyjne. ...}, Kraków 2002, s. 132}.
Zastrzegł, iż jest to definicja ogólna i wstępna, przewidując, że refleksja
nad sztuką Internetu będzie musiała zmieniać się wraz ze swoim przedmiotem.
Net art zmienia się szybko, równie szybko co technologia komputerowa,
nie mniej jednak ogólność definicji Ryszarda Kluszczyńskiego powoduje
jej aktualność. Według mnie, brak w niej podkreślenia związku net artu
z Internetem i technologią komputerową oraz niepotrzebny nacisk na
funkcjonowanie poza systemem galeryjnym, co przecież z biegiem czasu
może się zmienić.
Nieco późniejszą definicję sformułowała Ewa Wójtowicz. Określa ona net art
jako %Ewa
\textit{niematerialną, interaktywną sztukę, której podstawowym
medium i kontekstem jest Internet. Tematyka net artu najczęściej dotyczy
Internetu i komunikacji sieciowej oraz technologii. Projekt net art ma
hipertekstową, nielinearną budowę i zakłada aktywny odbiór w postaci
wielokierunkowej nawigacji. Oprócz wizualnej postaci dzieła, ważny jest
proces interaktywnej komunikacji. (...) Net artyści unikają definicji
i zacierają granice między sztuką a przestrzenią
Sieci}\footnote{ Ewa Wójtowicz, \textit{Net art}, Kraków 2008, s. 11-12}.
Jest to definicja dużo bardziej konkretna. Ścisły związek net artu
z Internetem i technologią jest bezpośrednio wyrażony wraz z kilkoma
istotnym cechami sztuki Internetu. Definicja pochodzi z roku 2008
i na razie (do roku 2010) w pełni się, według mnie, broni.
Na koniec chciałbym przytoczyć jeszcze obecną definicję net artu znajdującą
się w wikipedii. Znajduje się ona pod hasłem ,,Sztuka Internetu'' i ma
następującą postać: %Wikipedia
\textit{Sztuka, której głównym
medium, a najczęściej również tematem jest Internet. Istnieje tu podobieństwo
do sztuki wideo używającej przekazu wizualnego jako nośnika i jako tematu,
lecz z większym jeszcze naciskiem na temat niż na ów nośnik. Cytując definicję
stworzoną przez Steve'a Dietza, byłego kustosza działu nowych mediów
w Centrum Sztuki Walkera w Minneapolis: internetowe projekty artystyczne
to ,,projekty artystyczne, dla których sieć jest zarówno wystarczającym
jak i koniecznym warunkiem do zaistnienia.'' (...) Sztuka internetu może
przyjąć konkretną formę w postaci artystycznych stron www, sztuki w e-mail,
artystycznego oprogramowania internetowego, bazujących na internecie lub
połączonych w sieć instalacjach, online wideo, pracach radiowych lub audio,
sieciowych występach i instalacjach lub występach
offline}\footnote{ http://pl.wikipedia.org/wiki/Sztuka\_Internetu}.
Najbardziej odpowiadającą dla mnie jest definicja pani Ewy Wójtowicz,
która zawiera wszystkie najważniejsz cechy net art. Bardzo trafne jest
także określenie cytowane w wikipedii - net art to projekty, dla których
sieć jest zarówno wystarczającym jak i koniecznym warunkiem do zaistnienia.

%\chapter{Przykłady dzieł net art}
\section{Przykłady dzieł net art}
%Kilka przykładów dzieł net art, zarówno wczesnych jak i późniejszych.
%Chcę tutaj przybliżyć czytelnikowi temat zagadnienia net art poprzez
%konkretne przykłady.

%wstęp
Przedstawienie kilku przykładów zrealizowanych dzieł net art z pewnością
przybliży zagadnienie w najlepszy sposób. W tym miejscu chciałbym
opisać kilka realizacji, zarówno z początkowej fazy zjawiska jak
i z dojrzałego etapu, który trwa do dziś.

%The World's First Collaborative Sentence
Jednym z pierwszych przedsięwzięć,
które mogą zosać nazwane dziełem net art jest strona Davisa Douglasa
o nazwie \textit{The World's First Collaborative Sentence}, która działała
od roku 1994.
Każdy odwiedzający mógł dodać słowo do wielkiego,
powstającego wspólnym wysiłkiem, zdania. W ten sposób Davis chciał
pokazać możliwości drzemiące w komunikacji. Za pomocą klawiatury
komputera, który ma połączenie z Internetem każdy na świecie może
dołączyć do wspólnego tworzenia zdania. Takich możliwości nie ma
ani telewizja, ani video, dopiero Internet i komputer pozwalają artyście
na globalne interakcyjne dzieło. Praca artysty ewoluowała w czasie.
Wraz z rozwojem technologii do zdania można było dodawać nie tylko
słowa, ale również obrazki, linki i dźwięki oraz używać różnych krojów
i rozmiarów czcionek. Dzieło ewoluowało przez kilka lat wraz z Internetem.
Obecnie nie jest już dostępne, można znaleźć jego
opisy\footnote{ Douglas Davis, \textit{The World's First Collaborative
Sentence} (1994), http://stage.itp.nyu.edu/history/timeline/sentence.html}.

%The File Room
Niektóre pierwsze inicjatywy net art istniały jednocześnie w Internecie
i galerii wystawowej. Dotyczy to między innymi projektu
\textit{The File Room}. Jest to zbiór przypadków cenzury tekstów z całego
świata. W Internecie istnieje do dziś jako strona
internetowa\footnote{ Antonio Muntadas, \textit{The File Romm} (1994),
http://www.thefileroom.org/}. Odwiedzający
może dodać nowe wpisy za pomocą przygotowanego formularza. Może w ten sposób
przyczynić się do rozbudowania istniejącej bazy. Dzieło to zostało również
wystawione w galerii, gdzie goście mogli przeglądać zgromadzoną
bazę danych, jak również rozszerzać ją o znane im przypadki stosowania
cenzury.

%Jodi
W połowie lat dziewiędziesiątych powstało kilka grup artystów zajmujących
się sztuką Internetu. Jest to na przykład Jodi.org -- grupa
składająca się z dwóch artystów: Joan Heemskerk i Dirk Paesmans. Jedną
z ich pierwszych prac jest
\textit{wwwwwwwww.jodi.org}\footnote{ Jodi.org, http://wwwwwwwww.jodi.org/
(1995)}. Pozornie jest to strona internetowa, zawierającą bezsensowne
migające, zielone znaczki. Kiedy spojrzy się w źródło tej strony okazuje
się, że zawiera ono rysunki wykonane w technice ASCII-art. Sama strona
jest jednym wielkim odnośnikiem do innych stron tego projektu, w których
strutyrze można się zgubić. Podobnie jest w projekcie o nazwie
\textit{text.jodi.org}\footnote{ Jodi.org, http://text.jodi.org/}. Mamy tutaj
do czynienia z labiryntem kilkuset stron pełnych bezsensownych znaczków
i odnośników do kolejnych stron pełnych bezsensownych znaczków. Praca
ta może być postrzegana jako wskazanie na możliwość zgubienia się w Internecie
wśród mnóstwa stron, które nie posiadają tak naprawdę żadnej treści.

%IPPainting
Istnieją dzieła net art w których odbiorca bierze udział w współtworzeniu
dzieła w sposób świadomy, czasem przemyślany, a isnieją także takie,
które wykorzystują uczestnictwo odbiorcy w sposób nieświadomy lub informują
go o tym po fakcie. Przykładem może być projekt
\textit{IPPainting}\footnote{ http://www.b-l-u-e-s-c-r-e-e-n.net/ipPainting/},
w którym adres IP każdego odwiedzającego jest używany jako parametr w wielu
skomplikowanych algorytmach, które tworzą obrazy. Po wejściu na stronę
projektu każdy współtwórca może zobaczyć w tworzeniu jakich obrazów
wykorzystano jego adres oraz które dodane fragmenty są wynikiem jego odwiedzin.

%wstęp do haktywizmu
%http://en.wikipedia.org/wiki/Hacktivism
Późniejsze dzieła net art są często zaangażowane politycznie, społecznie
lub ekologicznie. Artyści niejednokrotnie parodiowali strony internetowe
korporacji lub instytycji, dokonywali wirtulanych porwań, plagiatów,
sabotaży i innego rodzaju ataków z użyciem technologii internetowych.
W spotkanej przeze mnie literaturze tego typu działania takie nazywa się
haktywizmem\footnote{ Ewa Wójtowicz, \textit{Net art}, str 126-127}.
Słowo to wywodzi się od angielskiego pojęcia \textit{hacking},
które media używają powszechnie na określenie łamania zabezpieczeń
systemów komputerowych w celu dokonania kradzieży lub innej szkody.
W subkulturze hakerów, na takie działania istnieje inne określenie --
\textit{cracking}\footnote{ http://catb.org/jargon/html/C/cracking.html}.
Haker, w przeciwieństwie do crackera, nigdy nie
powoduje żadnych szkód. Według nomenklatury społeczności hakerskiej
akcje, które powodują szkody serwisów, na przykład ataki denial-of-service
(ataki na usługę sieciową lub stronę internetową powodujące przeciążenie
serwera i uniemożliwienie korzystania z niej) nie powinny być nazywane
haktywizmem. Analogicznie do pojęcia crackingu można by je nazwać
cracktywizmem (choć pojęcia takiego nigdzie nie udało mi sie odnaleźć).
Podsumowując, net artyści zaangażowani społecznie, politycznie lub
ekologicznie posuwają się czasami do działań typowych dla crackerów.
Celem ich działań nie jest jednak tylko destrukcja lub chęć zysku,
lecz zwrócenie uwagi na ważny (z punktu widzenia autorów) problem
społeczny, polityczny lub ekologiczny. Dalej opisuję dwa przykłady
tego typu projektów.

%etoys (toywar)
W 1994 roku powstała w Szwajcarii grupa artystów \textit{Etoy}. Sposobem
i formami działania przypominają duża firmę, jednak treść ich przekazu jest
antykorporacyjna. w 1996 roku w ramach słynnej akcji \textit{Digital Hijack}
po wpisaniu takich słów jak: ,,Madonna'', ,,Playboy'', czy ,,Porsche''
w popularne wyszukiwarki, jednym z pierwszych wyników wyszukiwania była strona,
na której użytkownik widział jedynie migoczący tekst: ,,Zostałeś cyfrowo
porwany przez organizację Etoy''.
W 1995 roku grupa zarejestrowała adres internetowy www.etoy.com, którego
używa do dziś. 1 listopada 1999 została pozwana przez amerykańską firmę
sprzedającą zabawki przez Internet. Firma eToys postawiła zarzut
negatywnego wpływania na jej markę przez publikowanie na stronach grupy
treści anarchistycznych i pornograficznych. Sklep istnieje pod adresem
www.etoys.com, który zarejestrowany został w roku 1998, a więc trzy
lata później niż strona grupy. Decyzją sądu strona grupy została jednak
zablokowana. Spowodowało to przeprowadzenie wielu akcji skierowanych
przeciwko firmie eToys. Były to setki fałszywych zamówień, wielokrotne,
skoordynowane ataki denial-of-service czy też udostępnienie w Internecie
prywatnego numeru telefonu prezesa firmy. W wyniku przeprowadzonych
dywersji przez etoys i mnóstwa innych osób, które postanowiły pomóc grupie,
ceny akcji firmy zabawkarskiej spadły, w krótkim okresie czasu,
z 67 dolarów do 15\footnote{ http://toywar.etoy.com/}.
Wydarzenia te miały miejsce w gorącym okresie
przedświątecznym, co spowodowało bardzo znaczne obniżenie dochodów firmy.
Ostatecznym zwycięstwem było wycofanie pozwu i odblokowanie strony grupy
etoy. Akcja została nazwana najdroższym performance'em w historii i była
chyba pierwszym przypadkiem tak donośnego wpływu działań w Internecie
na działania w świecie realnym i na
odwrót\footnote{ Ewa Wójtowicz, \textit{Net art}, str 131-135}.

%0100101110101101.org (nikeground)
\textit{Nike Ground} jest przykładem akcji grupy 
0100101110101101.org podjętej w roku 2003 w Wiedniu. Latem tego roku
pojawiły się plany odnowy placu Karola. Zakładały one również ustawienie
wielkiego monumentu w kształcie loga firmy Nike, zmianę nazwy na
plac Nike oraz produkcję nowego modelu buta upamiętniającego to wydarzenie,
pierwszą parę miał założyć burmistrz miasta na uroczystości otwarcia
odnowionego placu
Nike\footnote{ http://pl.wikipedia.org/wiki/Nike\_(firma)\#Nike\_Ground}.
Powstała strona
internetowa\footnote{ 0100101110101101.org, \textit{Nike Ground}, 
http://www.0100101110101101.org/home/nikeground/website/index.html (2003)},
pawilon informacyjny, rozdano wiele ulotek informujących o planach.
Wszystko okazało się jednak fikcją, firma dementowała rozpowszechniane
informacje. Po jakimś czasie do zorganizowania akcji przyznała się
grupa 0100101110101101.org wraz z działającym w Wiedniu \textit{Institute
for New Culture Technologies}\footnote{ http://www.t0.or.at/nikeground}.
Firma złożyła pozew, który został później wycofany, z poowodu obaw
o stracenie dobrego wizerunku. Artyści chcieli zwrócić uwagę na problem
zawłaszczania powierzchni miejskich, a także na strategie marketingowe
stosowane przez firmę Nike. Oficjalne podsumowanie organizatorów
przedsięwzięcia brzmiało następująco: \textit{Próby zastraszenia, którymi
posłużył się koncern znany ze stosowania podobnych strategii marketingowych,
wróciły do niego niczym bumerang}\footnote{ Institute for New Culture
Technologies, \textit{Nike-Klage gegen Kunstprojekt zurückgezogen},
7 stycznia 2004 roku}.

%http://techsty.art.pl/HGS/ (powieść hipertekstowa)
Przykładem polskiej strony skupiającej inicjatywy net artowej jest
\textit{techsty.art.pl}\footnote{ http://techsty.art.pl/},
która skupia się głównie na hipertekście. Zawiera między innymi
hipertekstową\footnote{ http://techsty.art.pl/HGS/}. Fabuła powieści
jest różna, w zależności od tego, które odnośniki zostaną wybrane
przez czytelnika. Powieść można dzięki temu czytać wielokrotnie, za
każdym razem tworząc niejako nową fabułę.

%\chapter{Wpływy i historia}
\section{Wpływy i historia}
%Omówienie wpływów wcześniejszych ruchów na powstanie net art.
%Wskazać chcę przede wszystkim na:
%\begin{itemize}
%\item mail art (sztuka poczty (nieelektronicznej))
%\item konceptualizm
%\item sztuka kinetyczna
%\item performance
%\item sztuka wideo
%\item wielka awangarda
%\end{itemize}
%Dla każdego ,,kierunku'' chcę wskazać co wiąże go z net art,
%dlaczego jest uważany za ,,prekursora'' net art.

%mały wstępik
Pojawienie się net artu jest ściśle związane z pojawieniem się Internetu.
Techniki, mechanizmy i sposoby działania dzieł net art często nie są
jednak nowe, wywodzą się z kierunków i inicjatyw wcześniejszych.
W tym rozdziale chciałbym wskazać z jakich kierunków czerpie net art,
gdzie wcześniej pojawiły się cechy, które net art przyjął i niejednokrotnie
rozwinął.

%mail art
U podstaw sztuki Internetu (jak i Internetu w ogóle) leży komunikacja.
Podobnie jest ze sztuką poczty (ang. mail art), która zaistniała w latach
pięćdziesiątych XX wieku. Wykorzystywała ona dostępną powszechnie możliwość
komunikacji pocztowej. Artyści wysyłali rozmaite formy tekstowe, takie jak:
wiadomości, instrukcje, wiersze oraz pocztówki, fotografie, naklejki czy
rysunki. Adresatami były osoby znane lub nie, najczęściej nie oczekiwano
odpowiedzi, choć zdarzały się od tego wyjątki. Bardzo często proces
interakcji związany z przesyłanymi rzeczami (na przykład instrukcjami)
był ważniejszy od treści. Rozwój sztuki poczty był
spowodowany chęcią odcięcia się od polityki galerii artystycznych, które
ustanawiały rankingi artystów. Artyści szukali nowego, otwartego układu
sztuki, który może objąć zasięgiem cały świat, globalnej, otwartej,
wielokierunkowej komunikacji, za pomocą powszechnego
medium.\footnote{ Ryszard Kluszczyński,
\textit{Społeczeństwo informacyjne. ...}, Kraków 2002, s. 130-132}.
Pojawienie się Internetu, a wraz z nim net artu powoduje zamieranie sztuki
poczty, ponieważ wszystkie jej idee można realizować za pomocą nowego medium
w sposób szybszy, bardziej wydajny i dający dużo więcej możliwości.
%sporo jest jeszcze u Ewy, można rozwinąć ten wątek w razie potrzeby.

% konceptualizm (sztuka konceptualna)
Konceptualizm jest kierunkiem sztuki, który istniał od lat sześćdziesiątych
XX wieku. Artyści tego kierunku starali się uwypuklać proces twórczy
ponad wynik pracy twórczej. Najważniejsza jest tutaj idea, koncept.
Przedmiotu sztuki może w ogóle nie być. Widać tutaj daleko posuniętą
dematerializację w sztuce. Dzieła miały często postać happeningów
i performance'ów. Materialnych pozostałości po nich nie było, lub były
to formy dokumentalne: zdjęcia, filmy, przekazy świadków. Oprócz
dematerializacji bardzo istotne było również włączenie
czynnika czasu. Dzieło odbywało się w określonym czasie, później pozostawała
jedynie jego dokumentacja. Pomysł powstawał, rozwijał się w określonym czasie,
później następowała realizacja i dokumentacja. Czasem artyści pozostawali
na samym pomyśle, idei, bez dalszej realizacji.

W sztuce Internetu mamy najczęściej do czynienia z pełną dematerializacją
dzieła sztuki. Działa ono w postaci kodu, stworzonego przez artystę.
Nie możemy fizycznie dotknąć dzieła net artowego, a jednak jesteśmy w stanie
go doświadczać, odbierać za pomocą zmysłów (na razie nie wszystkich).
Co więcej, czas odgrywa tutaj ogromną rolę. Projekty nie są dostępne
w Internecie na zawsze, jedne znikają, inne się pojawiają. Dzieła przestają
istnieć w Internecie z woli autorów, z zapomnienia lub z zaniedbania (na
przykład z powodu nie zapłacenia za serwer lub nazwę domenową). W sztuce
Internetu formą dokumentacji zaistniałego dzieła mogą być opisy bądź kopie
działające w innych miejscach. Cześć projektów zamiera oczywiście bezpowrotnie,
bez kopii, ani żadnej innej formy dokumentacji, ale tak samo było na pewno
z niejednym dziełem sztuki konceptualnej.

Kilka inicjatyw net artowych bezpośrednio nawiązuje do sztuki konceptualnej.
W latach 1978-84 tajwański artysta
Tehching (Sam) Hsieh przeprowadził serię trwających po rok
performance'ów, których cykl nosi nazwę \textit{One Year Performance}.
Pierwszy z nich -- \textit{Cage Piece} (1978-79) -- polegał na roku
odosobnienia artysty w klatce własnej budowy. Przez określone godziny
performance był dostępny dla publiczności. Przez cały czas Tehching zachował
milczenie. Drugi performance, nazwany \textit{Time Piece}, rozegrał się
w latach 1980-81. Ze zdjęć robionych artyście regularnie w ciągu całego roku
powstał sześciominutowy film. W kolejnym -- \textit{Outdor Piece} (1981-82) --
Hsieh spędził rok na zewnątrz, w Nowym Jorku, nie chroniąc się w żadnym
budynku, samochodzie, ani nawet namiocie. W ramach czwatej części cyklu --
\textit{Art/Life} (1983-84) -- artysta razem z nieznaną mu wcześniej kobietą
(Lindą Montano) spędził rok czasu. Byli przywiązani do siebie za pomocą
sznura o długości 8 stóp
(2,4 metra)\footnote{ http://www.communityarts.net/readingroom/archivefiles/2002/09/year\_of\_the\_rop.php}.
Piąty performance -- \textit{No Art Piece} -- był rokiem bez sztuki,
w tym czasie (1985-86) Tehching nie rozmawiał na temat sztuki, nie widział jej,
nie tworzył, nie czytał o niej, nie wchodził do żadnych muzeów czy galerii.
Nawiązaniem do tego cyklu jest trzecia praca duetu MTAA o tytule
\textit{Update}. Dwie poprzednie prace o tym samym tytule były przeniesienie
na grunt Internetu dwóch innych dzieł konceptualnych. W ramach trzeciej
odsłony artyści udostępnili wiele krótkich filmów pokazujących
codzienne życie obu artystów w dwóch identycznych, niewielkich pomieszczeniach.
Czas spędzony na oglądaniu artystów jest rejestrowany dla każdego uczestnika.
Każdy kto wytrwa rok oglądając artystów otrzymuje jako nagrodę kod
projektu, dostając więc dzieło niejako na własność. Jest sporo różnic
w stosunku do oryginalnych performance'ów Tehchinga. Tutaj to widz musi
wykazać się cierpliwością, a nie artysta. Duet MTAA tłumaczy, że w ich wersji
nikt nie musi cierpieć, jednak przeniesienie znacznego ciężaru z artysty
na odbiorcę, można również odczytywać jako dostosowanie dzieła do cech
sztuki Internetu\footnote{ Ewa Wójtowicz, \textit{Net art}, Kraków 2008,
s. 151-153}.

%\footnote{ http://en.wikipedia.org/wiki/Tehching_Hsieh}

Kolejnym bardzo ciekawym, bezpośrednim nawiązaniem net artu do konceptualizmu
jest praca Matta Butlera \textit{XML Translation of Dan Graham's ,,Schema''},
która już w tytule nawiązuje do projektu Dana Grahama \textit{Schema}. 
Pomysł Grahama polegał na stworzeniu zestawu reguł dokumentu bez odniesienia
do jego treści. Butler zauważył tutaj uderzające podobieństwo do definicji
typu dokumentu (ang. Document Type Definition) definiującego strukturę
dokumentów XML. DTD jest rodzajem meta-języka, opisującego zasady poprawnego
tworzenia dokumentów XML, bez odniesienie się do ich treści. Jest więc tym
dla XMLa co Graham chciał stworzyć dla dowolnego dokumentu.
Butler przełożył zapis reguł, które zaproponował Dan Brawn na zapis w formacie
DTD\footnote{ http://www.mbutler.org/schema/}.

%sztuka kinetyczna (/ kinematyczna?)
Sztuka kinetyczna jest ściśle związana z ruchem. Mamy tutaj do czynienia
ze zmianą koncepcji dzieła jako fizycznego artefaktu na dzieło
jako wydarzenia, często efemeryczne, za każdym razem inne. Najbliższą sztuce
Internetu odmianą tego kierunku jest partycypacyjna sztuka kinetyczna.
Wydarzenie musi być w niej zainicjowane przez odbiorcę. To on (lub ona)
musi ,,uruchomić'' dzieło sztuki, aby wyzwolić ukryte w nim możliwości
i wziąć udział w wydarzeniu. Widzimy więc tutaj zaangażowanie
odbiorcy, procesualność i dzieło sztuki jako wydarzenie odbywające się
w określonym okresie czasu. Dzieło sztuki ma tutaj pewien rodzaj
autonomiii. Autor tworzy dzieło, które daje w ręcę odbiorców. Dzieło
zaczyna żyć swoim życiem, ponieważ odbiorcy bardzo często mają możliwości
zmian w tychże dziełach. Ciekawymi przykładami są tutaj niektóre prace
Lygii Clark: \textit{Bichos} i \textit{Obra mole} (ang. Miękkie dzieło).
W pierszej artystka oddaje do rąk odbiorców kilka zestawów
połączonych zawiasami płytek, które uczestnik może przemieszczać w ramach
przewidzianych dla nich ruchów. Druga wzpomniana praca jest swego rodzaju
rozwinięciem pierwszej. W ramach \textit{Obra mole} odbiorca również
wpływa na wygląd dzieła, może przemieszczać płytki względe siebie,
jednak efekt jest krótkotrwały. Po zakończeniu działania przez użytkownika,
płytki wracają do swojego oryginalnego położenia, sprawiając całkowitą
efemeryczność dzieła
sztuki\footnote{ Ryszard Kluszczyński, \textit{Sztuka interaktywna},
Warszawa 2010, s. 66-76}.

%sztuka działania (happening, performance)
Happeningi, performance i inne dzieła, które polegają na wykonywaniu
działania przy udziale bezpośrednim lub pośrednim publiczności
nazywane są sztuką działania. W happeningu może w bardzo łatwy sposób
dojść do zrównania pozycji artysty i odbiorcy. Artysta musi być otwarty
na wszelkie możliwe ruchy odbiorcy, który zaangażuje się w tworzenie
dzieła-wydarzenia wraz z nim. Happening nie może dojść do skutku
bez publiczności, podobnie jak każde inne dzieło sztuki działania.
Odnajdujemy więc tutaj uczestnictwo odbiorcy w tworzeniu dzieła,
zmuszenie artysty do adaptowanie się do zmiennych warunków i zachowań
odbiorców, a także czynnik czasu. Każde dzieło sztuki działania
jest wydarzeniem, które trwa w określonym czasie. Wymienione cechy
występują również w wielu projektach sztuki Internetu
\footnote{ Ryszard Kluszczyński, \textit{Sztuka interaktywna},
Warszawa 2010, s. 77}.

%sztuka wideo
Sztuka wideo to bardzo bliski krewny net art. W obu nurtach mamy do czynienia
ze sztuką opartą na medium, traktującą o nim, często drążącą graniczne
możliwości jego wykorzystywania. Przedstawię dwa dzieła zaliczające się
do tego gatunku. \textit{Opposing Mirrors and Video Monitors on Time Delay}
to instalacja Dana Grahama z 1974 roku w której uczestnik mógł spacerować
pośród luster i monitorów. Monitory przedstawiały obraz z kamer zainstalowanych
w instalacji opóźniony o kilka sekund. Uczestnik widział więc w lustrach
swoje aktualne odbicie oraz odbicia monitorów, które pokazywały obraz sprzed
kilku sekund, dająć możliwość widzenia swoich działań z różnych momentów czasu.
Instalacja \textit{Inter Nos} z roku 1977 chorwackiej artystki Sanji Iveković
to możliwość interpersonalnej komunikacji zapośredniczonej interfejsem
składającym się ze sprzęgnietych ze sobą kamer wideo i monitorów. Artystka
przebywając w jednym pomieszczeniu widziała na monitorze obraz rejestrowany
przez kamerę, znajdującą się w drugim pomieszczeniu, w którym znajdowali
się pozostali uczestnicy dzieła. Oni widzieli na monitorze w swoim
pomieszczeniu obraz artystki rejestrowany przez kamerę w pierwszym
pomieszczeniu. Skonstruowany system dawał możliwość komunikacji
w czasie rzeczywistym za pomocą obrazu i dźwięku. Była to właściwie
antycypacja wideokonferencji, które są dziś powszechnie odbywane.
Niektóre dzieła sztuki wideo próbowały wykorzystać technologię do
komunikacji, tworzyły dzieła interaktywne, zależne od uczestników
i od czasu, miały więc cechy sztuki Internetu, choć operowały innym medium,
inną technologią\footnote{ Ryszard Kluszczyński, \textit{Sztuka interaktywna},
Warszawa 2010, s. 105-109}.

%sztuka telematyczna
Dzieła telematyczne są ściśle związane z komunikacją i interkacją, a więc
z podstawami Internetu (i oczywiście sztuki Internetu). Rozkwit sztuki
telematycznej przypada na lata 80-te XX wieku. Najbardziej spektakularne
dzieła były reazlizowane za pomocą połączeń satelitarnych pomiędzy kilkoma
albo nawet kilkunastoma krajami, często z różnych kontentów. Połączeni
satelitarnie ze sobą artyści wykonywali ustalone czynności, np. tańczyli,
co było możliwe do zobaczenia na jednym ekranie, gdzie obraz miksowany był ze
wszystkich miejsc\footnote{ Ewa Wójtowicz, \textit{Net art}, Kraków 2008,
s. 175-177}.
%tu nie prawda !!!!!!!!
%Przykłady dzieł, które można zaliczyć do telematycznych
%można znaleźć również współcześnie. Jest to na przykład projekt tworzenia
%muzyki w jednym czasie przez kilku artystów znajdujących się w zupełnie
%różnych miejscach połączonych ze sobą zapomocą.
%\footnote{ http://en.wikipedia.org/wiki/Playing_for_change}
%\footnote{ http://www.playingforchange.com/}

%awangarda XX wieku czyli krótkie podsumowanie tematu nawiązeń
Wszystkie wymienione wyżej prądy i kierunki sztuki mają pewną wspólną cechę.
Wszystkie związane są pojęciem awangardy XX wieku. Jest to zespół trendów,
które odrzuciły dotychczasowe podejście do sztuki szukając nowego.
Awangarda ,,znajduje'' sztukę tam gdzie jej wcześniej nie było. Sztuka
zaczyna się przeplatać ze zwykłym życiem lub wykorzystywać środki, którymi
wcześniej gardziła, a które są wykorzystywane powszechnie do innych celów.
Podobnie jest ze sztuką Internetu, która wykorzystuje medium używane na co
dzień do rzeczy niezwiązanych zupełnie ze sztuką. Sztuka Internetu szuka
granic możliwości wykorzystania medium, tak jak awangarda szukała granic
wykorzystania innych rzeczy (w tym mediów takich jak na przykład wideo).
Awangarda zmieniła status odbiorcy, zmusiła go do działania poprzez
swoje iteraktywne dzieła, uwypukliła czynnik czasu, procesualność i zmienność.
Wszystkie te cechy zawiera również net art. Nie jest to wynalazek wzięty
z nikąd, wyrasta bezpośrednio z ruchów awangardowych XX wieku.

%wcześniejsze pitolenie
%Najpierw awangarda, potem sztuka konceptualna, mail art no i potem mamy
%net art. Kilka znaczących cech awangardy można znaleźć w net arcie:
%wzrost znaczenia odbiorcy, procesualność dzieł, czynnik czasu.
%Net art jest awangardowy, ponieważ szuka granicy możliwości nowego
%medium, tak jak awangarda szukała granic sztuki i nowych jej form.
%ewentualnie o modernizmie i postmodernizmie
%Czyli o tym, że Lev Manovic uważa, że net art jest związany z modernizmem,
%a Kluszczyński, że z postmodernizmem. Ogólnie to pojęcia modernizmu
%i postmodernizmu są niejasne i istnieje wiele ich różnych definicji
%i objaśnień, dlatego nie warto się zajmować tym temat w odniesieniu
%do sztuki internetu.

%\chapter{Cechy net art}
\section{Cechy net art}
%wstępik
W tym rozdziale chciałbym przedstawić cechy dzieł net artu.
Rozpocznę od tych, które są fundamentalne i przynależą do każdej
realizacji net artowej. Dalej opiszę te, które występują często,
ale nie są konieczne do uznania dzieła za część sztuki Internetu.

%hipertekst, nielinearność, nawigacja
Każde omówione wyżej i spotkane przeze mnie dzieło net art wykorzystuje
w jakiś sposób hipertekst. Pojęcie to wprowadzone zostało przez Teda
Nelsona w roku 1965 i oznacza możliwość odnoszenia się dokumentu
tekstowego do innego fragmentu tekstu w innym
dokumencie\footnote{ Ewa Wójtowicz, \textit{Net art}, Kraków 2008,
s. 106-107}. Wizja Nelsona obejmowała istnienie jednego zdecentralizowanego
dokumnetu zawierającego całość informacji. Projekt ten nazwany został
Xanadu\footnote{http://xanadu.com}, jednak jego złożoność nie pozwoliła
na wdrożenie i popularyzację.
W roku 1989 Tim Burners-Lee, fizyk, pracownik ośrodka badawczego CERN
zaproponował swoją wizję hipertekstowego systemu\footnote{Tim Berners-Lee,
\textit{Information Management: A Proposal.},CERN (March 1989, May 1990)}
opartego na istniejącej już wtedy sieci Internet. Projekt nazwany
zostaje \textit{World Wide Web} (W3). W 1991 roku pojawia się pierwsza
publiczna
wzmianka\footnote{http://lists.w3.org/Archives/Public/www-talk/1991SepOct/0003.html}
na temat języka znaczników HTML i programu do jego interpretacji - przegladarki
internetowej, które są głównymi częściami projektu W3.
Idee \textit{World Wide Web} przetrwały do dziś. Nadal używamy języka
znaczników HTML (choć mocno zmienionego i z wieloma dodatkami w porównaniu
do pierwowzoru) oraz przeglądarek internetowych. Autor stony internetowej
dzięki hipertekstowi może zbudować strukturę, przez którą użytkownik
porusza się według swojego własnego uznania. Autor hipertekstowej struktury
oddaje w ręce użytkownika możliwość nawigowania po stronach, nie daje on
jasno określonej, linearnej ścieżki przejść od jednej strony do następnej.
Jest wręcz przeciwnie, ścieżek może być wiele, dlatego mówimy, że hipertekst
jest nielinearny. Pozwala na wędrowanie po swojej strukturze zgodnie
ze znajdującymi się w nim odnośnikami, tworzącymi sieć, a nie linię
prostą\footnote{Ryszard W. Kluszczyński, \textit{Społeczeństwo informacyjne.
Cyberkultura. Sztuka multimediów}, Rabid, Kraków 2001, s. 64}. Istnieją
dzieła net art, które skupiają się na zagadnieniu
hipertekstu, na przykład wspomniana wyżej powieść hipertekstowa.
Inne hipertekst traktują jako narzędzie przekazu. W obu przypadkach
jest on jednak nieodłączną ich cechą.

%wirtulaność, inaczej niematerialność
Dzieło net art można zobaczyć, czasem również usłyszeć, jednak najczęściej
nie można dotknąć, powąchać ani posmakować. Są to rzeczy niematerialne.
lub mówiąc inaczej wirtualne. Pojęcie to może być jednak rozumiane na wiele
sposóbów. Tutaj używam go tak jak Pierre Levy, jako przeciwieństwo
aktualności. \textit{Wirtualność i aktualność są jedynie dwoma postaciami
rzeczywistości. [...] Wirtualność istnieje, choć jej tu nie
ma}\footnote{Pierre Levy, \textit{Drugi Potop}, przeł.Justyna Budzyk
[w:] \textit{Nowe media...}, s. 378-379}.
Większość przedsięwzięć sztuki Internetu
żyje tylko w medium, w Internecie. Niektóre mają wpływ również na
realność, jak na przykład działania w czasie trwania toy war, niektóre
istnieją równolegle w postaci niematerialnej strony i działań w realności
w formie happeningu lub innej -- tak było z projektem \textit{Nike Ground}
opisanym wcześniej. Każde dzieło net art musi zaistnieć w Internecie,
w cyberprzestrzeni, czyli w przestrzeni wirtualnej\footnote{Ewa, s. 109}.

%interakcja
Definicje interakcji, jej rodzaje i możliwe sposoby pojmowania są opisane
w następnym rozdziale tej pracy. Tutaj chciałem jednak zaliczyć już
interaktywność do cech, które występują w każdym dziele net art. Zauważmy,
że w Internecie to odbiorca decyduje jakimi ścieżkami się porusza,
decyduje co wpisuje do formularzy i gdzie klika. Interakcja jest tutaj
dwukierunkowa, system w miarę swoich zaprogramowanych możliwości może być
partnerem. W kierunkach, z których net art czerpie swoje cechy, na przykład
w sztuce kinetycznej, interakcja jest jednokierunkowa -- dzieło odpowiada
na działania odbiorcy-uczestnika. W sztuce Internetu dzieło może działania
inicjować, prowokować odbiorcę do reakcji, może nawet próbować przejmować
kontrolę. Widz zmienia sie w współtworzącego dzieło (czasem nieświadomie).
Wszystko to jest oczywiście iluzją mechanizmów stworzonych
przez autora dzieła, mamy tu jednak do czynienia z dwukierunkową interakcję.
Każde dzieło net art jest interaktywne, ponieważ każde wymaga aktywnego
udziału odbiorcy\footnote{Ewa, s. 115}.

%medium, technologia, wykorzystanie codziennego narzędzie jako tworzywo
%(no medium) sztuki
Sztuka Internetu, jak sama nazwa wskazuje jest nieodłącznie związana
z Internetem -- medium wykorzystywanym na co dzień przez miliony ludzi.
Służy im ono do pracy, zabawy i do codziennych spraw.
W tradycyjnej sztuce -- na przykład
malarstwie -- płótno, technika malarska czy specjalistyczne pędzle są
przypisane tylko i wyłącznie do tej dziedziny. W przypadku net artu
sztuka wchodzi w sferę codzienności, zaczyna korzystać z rzeczy powszechnych.
Nie jest to oczywiście nowatorstwo net artu, ponieważ sztuka dokonuje
tego już od dość dawna, tego typu przekształcenia pojawiły się w czasie
wielkiej awangardy. Warto tutaj zwrócić uwagę na fakt, że net
art wykorzystuje powszechnie dostępne i przez mnóstwo ludzi wykorzystywane
na co dzień mechanizmy i technologię -- jest częścią Internetu, jest
nieodłącznie z nim związany.
%\footnote(Sztuka interaktywna, s159-160} (od biedy ;))

%interludium
Kolejnych, opisanych dalej cech, nie można już przypisać do wszystkich
realizacji net artowych. Warto jednak zdawać sobie z nich sprawę, ponieważ
wiele dzieł sztuki Internetu je posiada.

%decentralizacja, komunikacja, temat
Wspomniałem wyżej, że powiązanie z Internetem jest nieodłącznym aspektem
net artu. niektóre dzieła potengują ten związek poprzez obranie za
swój temat medium w którym występują. Istnieje wiele dzieł sztuki
Internetu, które poruszają zagadnienia komunikacji, próbują analizować
wpływ Internetu jako nowego medium na życie ludzi, próbują również
badać możliwości jakie ono daje, próbują wykorzystywać medium w sposób
na który nikt wcześniej nie wpadł badając w ten sposób jego granice.
Za przykład mogą posłużyć tutaj opisane wcześniej projekty grupy Jodi.

%zaangażowanie polityczne/społeczne/ekologiczne
Dzieła net art, których tematyka związana jest z medium, występowały
w największej liczbie we wczesnych latach tego kierunku. Internet
był wtedy jeszcze ,,młody'', artyści starali się go poznać. W późniejszych
latach zaczęły pojawiać się realizacje odnoszące swój temat do ważnych
kwestii społecznych, politycznych bądź ekologicznych. Przykładami są tutaj
omówione wcześniej akcje \textit{toy war} i \textit{nikeground}.

%czas
Czynnik czasu został wprowadzony do sztuki jako element pierwszej
wagi w czasach Wielkiej Awangardy. Sztuka Internetu w większy lub mniejszy
sposób akcentuje ten czynnik. Większość dzieł net art zmienia się 
wraz z upływem czasu, ponieważ zmienia je interakcja z odbiorcami-uczestnikami.
Wiele dzieł przy każdym odbiorze jest innych -- każda interakcja powoduje
zmianę.
Innyn aspektem jest nietrwałość i przemijalność dzieł tego kierunku.
Wraz z bardzo szybkim rozwojem technologii, stare realizacje net artu
stają się nieprzystosowane. Dzieło sztuki powinno być aktualizowane
i odświeżane, aby nadażyć za zmiana. Dodatkowo za każdą nazwę domenową
należy często corocznie uiścić opłaty. Brak zapłaty oznacza zniknięcie
strony z Internetu. Może ona istnieć nadal na innym serwerze, jednak nie
będzie pod oryginalnym adresem. Obecnie istnieje bardzo niewiele dzieł
net art, które rozpoczęte w początkach istnienia tego kierunku działają
po dziś dzień. Rozwój technologii jest bezlitosny i bardzo szybki.

%decentralizacja
Decentralizacja to ostatnia cecha, na którą chciałbym zwrócić uwagę
w tym rozdziale. Jednym z fundamentów Internetu jest jego decentralizacja.
To dzięki temu mógł się on rozrosnąć do obecnych, nieprzewidzianych
nawet przez autorów, rozmiarów. Większość dzieł i serwisów internetowych
mieści sie na jednym serwerze, są jednak takie realizacje, które chcą
zwrócić uwagę na zdecentralizowanie Sieci. Przykładem może być historia
\textit{Agatha Appears} Olii Lialiny z roku 1997. Początek opowieści
znajduje się na serwerze węgierskiego instytutu medialnego. Dalsze części
znajdują się na innych serwerach w różnych krajach. Same adresy URL są
również częscią projektu, ponieważ wskazują kolejne losy
bohaterki\footnote{Ewa, str 34-35}:
\begin{verbatim}
http://www.here.ru/agatha/cant_stay_anymore.htm
http://www.altx.com/agatha/starts_new_life.html
http://www.distopia.com/agatha/travels.html
http://www2.arnes.si/~ljintima3/agatha/travels_a_lot.html
http://www.zuper.com/agatha/wants_home.html
http://www.ljudmila.org/~vuk/agatha.goes_on.html
\end{verbatim}

%otwartość, powszechność, bezgraniczność (kluszczyński)
% to są brednie niepoparte żadnymi faktami!!!
%Z decentralizacją związane są również otwartość, powszechność
%i bezgraniczność dzieł sztyki Internetu. Wydaje się, że kod każdej strony
%można podejrzeć, skopiować oraz przenieść na inny serwer (co nie zawsze jest
%legalne). Nie zawsze jednak jesteśmy w stanie skopiować mechanizm generowania
%stron internetowych, ponieważ może być on zaszyty na serwerze. Niektórzy
%artyści kierunku związani są z ruchem Wolnego Oprogramowania, głoszącego
%idee otwartości i powszechności kodu źródkłowego programów. Podobnie jest
%z dziełami tychże artystów -- ich kod jest publicznie dostępny

%Podsumowanie
% ????

%\chapter{Interaktywność}
\section{Interaktywność}
%różne definicje interaktywności, kilka jest w ewie, m.in od
%kluszczyńskiego (cały rozdział na temat interaktywności
%w społeczeństwie...) definicja z netopedii, wikipedii, i co sie znajdzie
W tym rozdziale przeanalizuję pojęcie interaktywności, począwszy od
różnych definicji, poprzez klasyfikację i podziały, a kończąć na głosach
krytycznych.

Badam interaktywność w kontekście sztuki Internetu, dlatego
w pierwszej kolejności przytoczę definicję z kilku słowników i leksykonów
występujących w medium tej sztuki. Najpopularniejszym obecnie słownikiem
w Interneci jest wikipedia\footnote{http://www.wikipedia.org/}, która
w polskiej wersji językowej definiuje, że 
\textit{Interaktywność (ang. interactivity, łac. interactus  – dosłownie:
czyn wzajemny) to pojęcie z dziedziny komunikacji. Oznacza zdolność do
 wzajemnego oddziaływania na siebie przez komunikujące się strony.
Pojęcie interaktywności najczęściej stosowane jest w informatyce, telewizji
i multimediach. W tym kontekście, ma za zadanie podkreślić zdolność programu
lub urządzenia do jednoczesnego odbierania informacji i reagowania na
nią}\footnote{http://pl.wikipedia.org/wiki/Interaktywność}.

Inny internetowy leksykon -- Netopedia\footnote{http://webstyle.pl/netopedia}
-- definiuje interaktywność w następujący sposób:
\textit{Bezpośrednia wymiana informacji między komputerem (programem,
stroną WWW) a człowiekiem. Systemy interaktywne przystosowane są do
prowadzenia dialogu z obsługującymi je ludźmi - odbierania wprowadzanych
przez nich danych lub reagowania na komendy sterujące. Interaktywne są
wszystkie aplikacje użytkowe - edytory tekstów, arkusze kalkulacyjne,
programy graficzne, itp.}\footnote{http://webstyle.pl/netopedia/spoleczenstwo\_informacyjne/interaktywnosc}.

Interaktywność można opisywać również w kilku płaszczyznach tak jak Martin
Lister:
\textit{Technicznie -- zdolność do interwencji przez użytkownika w procesy
obliczeniowe komputera i możliwość zobaczenia efektów tej interwencji
w czasie rzeczywistym. W teorii komunikacji opisuje komunikację międzyludzką
opartą na dialogu i wymianie}\footnote{Martin Lister (i in.),
\textit{New Media...}, s. 388}.
Mamy tutaj wskazaną płaszczyznę czystko techniczną oraz płaszczyznę
komunikacji międzyludzkiej.%do zmiany

Ryszard Kluszczyński opisuje interaktywność na różne sposoby.
\textit{W najogólniejszym ze stosowanych znaczeń termin interaktywność
funkcjonuje jako określenie charakteru pewnego typu relacji między
przedmiotem i jego użytkownikiem. Pozostający w tej relacji obiekt ujawnia
swoje właściwości, oraz, co najważniejsze, wypełnia swoje funkcje wówczas
jedynie, gdy użytkownik zachowuje się w sposób aktywny, gdy wykorzystuje
obiekt jako nadzędzie realizacji swoich dążeń. Te ostatnie, rzecz jasna,
muszą pozostawać w zgodzie z konstrukcją i przeznaczeniem
obiektu}\footnote{Ryszard W. Kluszczyński,
\textit{Społeczeństwo informacyjne}, s. 96}.

Natomiast w innej książce przyjmuje następującą definicję.
\textit{Interaktywność to własność dzieła sztuki, która sprawia, że jego
odbiorca może podjąć działania, które wpłyną na kształt końcowy dzieła.
Stanowi jedną z najważniejszych cech współczesnej kultury. Interaktywność
w sztuce jest dialogiem odbiorcy (interaktora) i artefaktu w czasie
rzeczywistym, komunikacją w formie wzajemnego oddziaływania. Sprawia,
że dzieło jest inne w każdym przypadku aktywności odbiorcy
(interaktora)}\footnote{http://pl.wikipedia.org/wiki/Interaktywność\_w\_sztuce}
\footnote{Ryszard W. Kluszczyński, \textit{Film-Wideo-Multimedia. Sztuka
ruchomego obrazu w erze elektronicznej.} Instytut Kultury Warszawa 1999}.
Definicja ta znajduje się również w wikipiedii pod hasłem interaktywność
w sztuce.

%poziomy interkacji według Simsa (reaktywna, koaktywna, proaktywna)

%4 typy interaktywności według Zimmermana

%krytyczne podejście do interaktywności, prace wskazujące na jej
%ograniczenia czy wręcz iluzoryczność

%podsumowanie rozdziału, m.in. że nie wszystko co interaktywne
%jest net artem, ale wszystko co net artem to interaktywne

%wiele cennych uwag na temat interkaktywności w sztuce internetu
%jest w ewie
%wiele cennych rzeczy nadających się do zakończenia (ostatniego rozdziału)
%jest w 5-tym rozdziale Społeczeństwa... Kluszczyńskiego.

%\chapter{Czy net art jest interaktywny}
\section{Czy net art jest interaktywny}
%Konfrontacja net art i różnych definicji interaktywności.
%Próba odpowiedzi napytanie czy każde dzieło net art jest
%interaktywne.

%\section{Zakończenie}
%Zakończenie.

\begin{thebibliography}{6}
%\footnote{ Ewa Wójtowicz, \textit{Net art}, Kraków 2008, s. 1-2}
\bibitem{EWA}
  Ewa Wójtowicz:
  \textit{Net art},
  Wydawnictwo Rabid, Kraków 2008
%\footnote{ Ryszard Kluszczyński, \textit{Sztuka interaktywna}, Warszawa 2010, s. 1-2}
\bibitem{SZT-I}
  Ryszard W. Kluszczyński:
  \textit{Sztuka interaktywna},
  Wydawnictwa Akademickie i Profesjonalne,
  Warszawa 2010
\bibitem{MARYLA}
  praca wspólna pod red. Maryli Hopfinger:
  \textit{Nowe media w komunikacji społecznej w XX wieku}
  Oficyna wydawnicza,
  Warszawa 2002
\bibitem{E-I-AE}
  Maria Gołaszewska:
  \textit{Estetyka i antyestetyka},
  Wiedza Powszechna, Warszawa 1984
%\footnote{ Ryszard Kluszczyński, \textit{Społeczeństwo informacyjne. ...}, Kraków 2002, s. 1-2}
\bibitem{CYBERKULTURA}
  Ryszard W. Kluszczyński:
  \textit{Społeczeństwo informacyjne. Cyberkultura. Sztuka multimediów},
  Wydawnictwa Rabid, Kraków 2002
\bibitem{ESTETYKA-WIRTUALNOSCI}
  praca wspólna pod red. Michała Ostrowickiego:
  \textit{Estetyka Wirtualności},
  Towarzystwo Autorów i Wydawców Prac Naukowych UNIVERSITAS,
  Kraków 2005
\bibitem{MCLUHAN}
  Marshall McLuhan:
  \textit{Wybór tekstów},
  red. Eric McLuhan, Frank Zingrone,
  przeł. Ewa Różalska, Jacek M. Stokłosa
  Zysk i S-ka Wydawnictwo, Poznań 2001
\bibitem{VIDEO-W-POLSCE}
  Ryszard W. Kluszczyński:
  \textit{Zarys historii sztuki wideo w Polsce},
  [w:] Culture.pl, 28 sierpnia 2010,
  http://www.culture.pl/pl/culture/artykuly/es\_historia\_wideo
%tu nie ma praktycznie nic na temat net art!
\bibitem{OBRAZY-NA-WOLNOSCI}
  Ryszard W. Kluszczyński:
  \textit{Obrazy na wolności. Studia z zakresu sztuk medialnych w Polsce.},
  Instytut Kultury, Warszawa 1998
\bibitem{GALAKTYKA-INTERNETU}
  Manuel Castells:
  \textit{Galaktyka Internetu},
  Wydawnictwo Rebis, Poznańm 2003
\bibitem{ERIC-ZIMMERMAN}
  Eric Zimmerman:
  \textit{Narrative, Interactivity, Play, and Games: Four naughty concepts in need of discipline},
  [w:] Wardrib-Fruin, Noah, Pat Harrigan (red.),
  \textit{First Person. New Media as Story, Performance, and Game.},
  The MIT Press, London,
  http://www.ericzimmerman.com/texts/Four\_Concepts.html

\end{thebibliography}

\end{document}

% vi:set encoding=utf8:
